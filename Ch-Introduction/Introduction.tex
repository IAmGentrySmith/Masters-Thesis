\chapter{Introduction}
\label{ch:Introduction}

I want a first paragraph that inspires the reader to continue reading. Maybe something with a quote or a question. Maybe not.

The idea for an introduction is to interest the reader and provide general background information. 
For my work, the interesting part is how impactful computational science (specifically chemistry) has been in research and on society. 
The background information will start extremely generic and then go into some overarching themes. 
Generically, I'll include the development of computational science in the 20th century and the vast applications of computational chemistry specifically. 
Overarching themes are related, but don't have to be explicitly relevant. 
For example, Levinthal's paradox is a fun example showing the problem of conformation landscape searches.
Also, 

\section{Reason for Study}

The studies conducted that comprise this work were determined as a combination of collaborative efforts and larger research group goals with novel discoveries worth reporting.
Collectively, they explore conformations of internal bond dihedrals, molecule orientations of microstates, and properties of interfacing substrates.
These efforts are categorized and separated into three categories: ice crystal states, conformation landscapes, and ice interfaces. 
A brief introduction of each and a literature review of relevant information is given below.

\subsection{Generation of Ice I$_{h}$ Crystal Structure}

Ice crystals can take many forms based on properties like temperature and pressure. 
The proton-disorganized orthorhombic form of ice known as I$_{h}$ is the form of ice most commonly found on earth (general understanding, CITATION NEEDED). 
Due to the inherent randomness of the disorganization of the molecules within the crystal, computational efforts are limited in scope or instead utilize the proton-ordered orthorhombic form of ice XI  (general understanding, CITATION NEEDED).
This project explores a method to produce a high quality pseudorandom ice I$_{h}$ crystal structure.

\subsection{Conformation Landscapes of Group IV Chains}

Any molecule with a chain length of at least four contains at least one dihedral. 
In small molecules, the steric hindrance between the head and tail atoms are usually minimized in the fully gauche conformation to produce the lowest-energy conformer. 
In larger and more bulky molecules, additional interactions may cause the dihedral to take other conformations in search of the lowest-energy conformation. 
This project details the search for the lowest-energy conformer of a bulky hexagermane molecule in collaboration with Oklahoma State University's Charles Weinert and the complications and curiosities found within.

\subsection{Ice Interfaces in Two Dimensions}

Ice is capable of interfacing with hydrophobic and amphipathic molecules. 
These interfaces often have a specific range of polarity and \textbf{physicochemical} properties like atom type, charge, and the spacing of interfaces.
While usually considered in the biochemical sense, molecular interfaces with water have additional applications. 
If, for example, an interface had the properties to encourage ice growth, then it may be possible to observe significant ice growth above the expected freezing temperature. 

Modeling water systems can be computationally intensive and require significant resources or a limited scope of system.
One proven method of simplifying water modeling is to reduce the system by one dimension, creating a two-dimensional water.
The 'Mercedez Benz' BN2D model and the more recent rose potential model are two examples of two-dimensional water models.
Exploring and studying a two-dimensional water system allows for computationally-efficient investigating and can provide relevant information for a similar simulation with three-dimensional water.
This project covers the usage of the two-dimensional rose potential model to observe ice growth on a substrate with varying properties.

\section{Ice Annealing}

EACH SUBSECTION: DEFINITION OF TERMS

\subsection{Bernal - Fowler Ice Rules}

Citation! \cite{BFIceOG}

There have been many works published on this topic, so I will have no problem obtaining a cohesive review. I plan to include the various forms of ice and focus on I$_{h}$ and XI. This includes differences between the two in prevalence, environment, and synthesis (is synthesis the correct term here?). 

I expect to focus on many thermodynamic properties and to focus on entropy as a driving force of difference in the modeled system. I will include publications of the effort to model a truly proton-disordered system and focus on the computational aspect. 

\subsection{Structures of Ice}

Ice contains many structures.
These structures are typically orthorhombic, which means rectangular at non-90º angles.
They can also be hexagonal, cubic, else. 
These different structures form based on temperature and pressure. 
Ice I$_{h}$ is the most prevalent form of ice found normally on earth, forming at pressures around 1 atm and temperatures around 0ºC. (source needed?).

More detail on ice I$_{h}$.

More detail on ice XI.

\subsection{Residual Entropy of Ice I$_{h}$ and Ice XI}

Residual entropy is a thermodynamic property that greatly differs between ice I$_{h}$ and XI. 
In general, entropy can be calculated for a system of $N$ molecules as $S = Nk\ln(w)$, where $k$ is the Boltzmann constant and $w$ is the number of real microstates corresponding to any macrostate.
Residual entropy differs in calculation from entropy in that it generally refers to the entropy of a crystal near zero kelvin. 
Linus \cite{PaulingIce} described the $w$ of very low temperature crystals to approach the number of orientations possible for each molecule with consideration to immediate neighbors. 
Since ice I$_{h}$ is a 

For a proton-disordered ice I$_{h}$ crystal, $w$ becomes $\frac{x}{2^{4}}$ where $x$ is the number of acceptable orientations within the crystal.

\subsection{Hydrogen Bond Defects in Ice Crystals}

Speak here about defects and how they quantitatively harm stability and why the defects need to be reduced in general ice I$_{h}$ structures.

\subsection{Literature Review on Relevant Works}

\section{Conformation Landscapes}

EACH SUBSECTION: DEFINITION OF TERMS

For ROUGHLY forty years, computational programs have allowed investigators to model chemical systems with high accuracy to determine their physical properties.

\subsection{A Brief History of Conformation Landscapes}

\subsubsection{Levinthal's Paradox}

Discuss history of Levinthal and his paradox. Provide the non-paradoxical solution.

Next: Levinthal golf courses by Ken Dill.

\subsection{Computational Modeling}

Introduce importance and impact. Bigly important.

\subsubsection{History of Chemical Modeling}

Here I will introduce molecular modeling. This will begin with a brief history of the development of the field. It will continue through to mention the styles and goals of molecular modeling. Upon reaching modern techniques, I will discuss the benefits and costs associated with the major types of calculations (QM, MD, MC, etc). 

\subsubsection{Hardware: Oklahoma State University's Cowboy Cluster}

It would also be appropriate to mention the computational capabilities of OSU's Cowboy cluster.

\subsubsection{Software}

Software to mention: 

Visualizing{Avogadro, UCSF Chimera(?)}, 

Computing{GAMESS(?), Gaussian, NWChem(?), OOPSE}

This also includes brief pros and cons about the programs and the general purpose of use in case.

\subsubsection{Programming Languages}

Another hugely important portion of this will include a choice in the programming languages used (mostly python (Cython-compiled!), some Perl and Bash).

\subsection{Literature Review on Relevant Works}

\section{Modeling Germanium Compounds}

EACH SUBSECTION: DEFINITION OF TERMS

This will be an interesting section as there is extremely little in terms of Ge computational work. Perhaps a broader search will yield interesting results. For sake of thoroughness, I will also include work on computational energy optimization in general and work through complications brought by the size of Ge. I might also include a portion on the statistical spread of conformations at a given temperature (internal energy?) I may include a sentence or paragraph on Gaussian-based publications. 

\subsection{Tools for Modeling Germanium}

Computational Requirements and reasons for those requirements.

Germanium is not the most-studied atom in computational works.
The majority of Germanium studies are done with Gaussian (citation needed?).

\subsection{Computational Complexity of Germanium Compounds}

Draw-backs of modeling Germanium. Uncommon but still necessary for wetwork.

\subsection{Literature Review on Relevant Works}

Make note of various Germanium modeling research. Make note of tools and methods used.

\section{Two-Dimensional Water}

EACH SUBSECTION: DEFINITION OF TERMS

Rose water is fairly new on the computational scene and so I may also include a review on the Mercedez-Benz water system as well as any other attempts to model water in two dimensions. For the rose potential system, I will review the Lennard-Jones potential  as well as any other equations/systems related to the rose potential.

\subsection{Lennard-Jones Potential}

The Lennard-Jones Potential well is a soft-sphere model of interaction  between two spheres described with

\begin{equation}
V_{LJ} = 4\epsilon[(\frac{\sigma}{r})^{12} - (\frac{\sigma}{r})^{6}]
\end{equation}
where $V$ is the potential, $r$ is the distance between the center of two particles , $\sigma$ is the specific distance between the two particles where the potential is zero, and -$\epsilon$ is the minimum potential of the plot.
\textbf{REFINE WORDING:}
The plot is defined in $[0,\infty)$.
As two particles approach from infinity, their interaction become negative - which is an attractive force - and will approach the global minimum of -$\epsilon$.
The $r$ of this interaction is slightly larger than the combined radii of the two particles - which means they aren't quite touching - and is the equilibrium distance between the two particles.
As $r$ decreases beyond the minimum and toward $\sigma$, the interaction strength increases and reaches zero as $r = \sigma$.
At $r < \sigma$, 

Potential digression:
In a "hard-sphere" model, a particle's radius is firm, which is to say that the interaction potential is infinite at $r$ less than $\sigma$. 
Basically a vertical line between two discrete values (usually $\epsilon$ and $\infty$)as the potential shifts from $r \geq \sigma$ to $r < \sigma$ (maybe include image?).
The Lennard-Jones potential is a "soft-sphere" model, which blurs the line and replaces the vertical line with a functional representation. 
This breaks with reality as the particles become "squishy" and the potential ramps up toward infinity as $r$ decreases. 
The benefit to the soft-sphere model is that modeling programs can more-easily account for overlaps in particles during time steps than with hard-sphere models. 
For example, a hard-sphere model of two particles interacting will likely not have a position where $r = \sigma$ and will potentially overlap. 
At this overlap, the potential is infinity and will introduce a nearly infinite force at that instant of time. 
Computer systems do not like having infinitely large repulsions suddenly introduced into a simulation. 


\subsection{Modeling Water in Two Dimensions}

Modeling in two dimensions sacrifice the "realism" of models in three dimensions, but reduce the computational load significantly.
This allows researchers (scientists, chemists, digital magicians?) to test more simple designs in two dimensions as well as a higher volume of simulations at the same time/computational cost. 


\subsubsection{Mercedes-Benz Model}

The "Mercedes-Benz" BN2D model of water first proposed by \cite{MBWater} as "waterlike particles" are a popular two-dimensional representation of water. 
ROUGH:
details of shape of MB water

The mathematical model used in the BN2D model is generated from the Percus-Yevick equation by substituting the approximation

\begin{equation}
c(X_{1}, X_{2}) = y(X_{1}, X_{2})f(X_{1}, X_{2})
\end{equation}

into the Percus-Yevick equation obtained from the Ornstein-Zernike relation 

\begin{equation}
h(X_{1}, X_{2}) = c(X_{1}, X_{2}) + \frac{\rho}{2\pi}\int c(X_{1}, X_{3}) h(X_{3}, X_{2})dX_{3}
\end{equation}

to produce the overall relation

\begin{equation}
y(X_{1}, X_{2}) = 1 + \frac{\rho}{2\pi}\int y(X_{1}, X_{3})f(X_{1}, X_{3}) \times \Big[ y(X_{3}, X_{2})f(X_{3}, X_{2}) + y(X_{3}, X_{2}) - 1 \Big] dX_{3}
\end{equation}


\subsubsection{Rose Potential Model}

The rose potential is another model first introduced by \cite{RoseOG}.
This model, while similar to the three-pronged BN2D, is notably different in that the rose potential model simplifies the model by use of a radial sinusoidal plot to make the three "prongs" of the particle. 

\subsubsection{Two-Dimensional Modeling}

Something other than OOPSE? (not seeing obvious answer other than "custom code modified/forked from existing 3D tools")

The Object Oriented Parallel Simulation Engine (OOPSE) was introduced by \cite{OOPSE} as a relatively light-weight molecular dynamics simulation package focused on "efficiently integrating equations of motion for atom types with orientational degrees of freedom" (from abstract).
While OOPSE was further developed and renamed OpenMD, a fork of OOPSE was developed specifically to model water in two dimensions.

\subsection{Literature Review on Relevant Works}

TODO:
Work with MB water,





