\chapter{Introduction}
\label{ch:Introduction}

\section{First Words}

For nearly a century, computational chemistry has greatly assisted wet lab research and discovery and has relatively recently become its own field of focus within chemistry.


\section{A Brief History of Computational Chemistry}

The section is a work in progress and will be expanded upon during the next week.













I want a first paragraph that inspires the reader to continue reading. Maybe something with a quote or a question. Maybe not.

The idea for an introduction is to interest the reader and provide general background information. 
For my work, the interesting part is how impactful computational science (specifically chemistry) has been in research and on society. 
The background information will start extremely generic and then go into some overarching themes. 
Generically, I'll include the development of computational science in the 20th century and the vast applications of computational chemistry specifically. 
Overarching themes are related, but don't have to be explicitly relevant. 
For example, Levinthal's paradox is a fun example showing the problem of conformation landscape searches.
Also, 

\section{Reason for Study}

The studies conducted that comprise this work were determined as a combination of collaborative efforts and larger research group goals with results and discoveries worth reporting.
Collectively, they explore conformations of internal bond dihedrals and molecule orientations of microstates.
These efforts are categorized and separated into sections.
% WHAT CATEGORIES
A brief introduction of each and a literature review of relevant information is given below.

\subsection{Generation of Ice I$_{h}$ Crystal Structure}

Ice crystals can take many forms based on properties like temperature and pressure. 
The proton-disorganized orthorhombic form of ice known as I$_{h}$ is the form of ice most commonly found on earth (general understanding, CITATION NEEDED). 
Due to the inherent randomness of the disorganization of the molecules within the crystal, computational efforts are limited in scope or instead utilize the proton-ordered orthorhombic form of ice XI  (general understanding, CITATION NEEDED).
This project explores a method to produce a high quality pseudorandom ice I$_{h}$ crystal structure.

\subsection{Conformation Landscapes of Group IV Chains}

Any molecule with a chain length of at least four contains at least one dihedral. 
In small molecules, the steric hindrance between the head and tail atoms are usually minimized in the fully gauche conformation to produce the lowest-energy conformer. 
In larger and more bulky molecules, additional interactions may cause the dihedral to take other conformations in search of the lowest-energy conformation. 
This project details the search for the lowest-energy conformer of a bulky hexagermane molecule in collaboration with Oklahoma State University's Charles Weinert and the complications and curiosities found within.

