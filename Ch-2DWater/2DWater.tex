\chapter{Crystal and Liquid 2D Water at Interfaces}
\label{ch:2DWater}


\section{Two-Dimensional Water}

EACH SUBSECTION: DEFINITION OF TERMS

Rose water is fairly new on the computational scene and so I may also include a review on the Mercedez-Benz water system as well as any other attempts to model water in two dimensions. For the rose potential system, I will review the Lennard-Jones potential  as well as any other equations/systems related to the rose potential.

\subsection{Lennard-Jones Potential}

The Lennard-Jones Potential well is a soft-sphere model of interaction  between two spheres described with

\begin{equation}
V_{LJ} = 4\epsilon[(\frac{\sigma}{r})^{12} - (\frac{\sigma}{r})^{6}]
\end{equation}
where $V$ is the potential, $r$ is the distance between the center of two particles , $\sigma$ is the specific distance between the two particles where the potential is zero, and -$\epsilon$ is the minimum potential of the plot.
\textbf{REFINE WORDING:}
The plot is defined in $[0,\infty)$.
As two particles approach from infinity, their interaction become negative - which is an attractive force - and will approach the global minimum of -$\epsilon$.
The $r$ of this interaction is slightly larger than the combined radii of the two particles - which means they aren't quite touching - and is the equilibrium distance between the two particles.
As $r$ decreases beyond the minimum and toward $\sigma$, the interaction strength increases and reaches zero as $r = \sigma$.
At $r < \sigma$, 

Potential digression:
In a "hard-sphere" model, a particle's radius is firm, which is to say that the interaction potential is infinite at $r$ less than $\sigma$. 
Basically a vertical line between two discrete values (usually $\epsilon$ and $\infty$)as the potential shifts from $r \geq \sigma$ to $r < \sigma$ (maybe include image?).
The Lennard-Jones potential is a "soft-sphere" model, which blurs the line and replaces the vertical line with a functional representation. 
This breaks with reality as the particles become "squishy" and the potential ramps up toward infinity as $r$ decreases. 
The benefit to the soft-sphere model is that modeling programs can more-easily account for overlaps in particles during time steps than with hard-sphere models. 
For example, a hard-sphere model of two particles interacting will likely not have a position where $r = \sigma$ and will potentially overlap. 
At this overlap, the potential is infinity and will introduce a nearly infinite force at that instant of time. 
Computer systems do not like having infinitely large repulsions suddenly introduced into a simulation. 


\subsection{Modeling Water in Two Dimensions}

Modeling in two dimensions sacrifice the "realism" of models in three dimensions, but reduce the computational load significantly.
This allows researchers (scientists, chemists, digital magicians?) to test more simple designs in two dimensions as well as a higher volume of simulations at the same time/computational cost. 


\subsubsection{Mercedes-Benz Model}

The "Mercedes-Benz" BN2D model of water first proposed by \cite{MBWater} as "waterlike particles" are a popular two-dimensional representation of water. 
ROUGH:
details of shape of MB water

The mathematical model used in the BN2D model is generated from the Percus-Yevick equation by substituting the approximation

\begin{equation}
c(X_{1}, X_{2}) = y(X_{1}, X_{2})f(X_{1}, X_{2})
\end{equation}

into the Percus-Yevick equation obtained from the Ornstein-Zernike relation 

\begin{equation}
h(X_{1}, X_{2}) = c(X_{1}, X_{2}) + \frac{\rho}{2\pi}\int c(X_{1}, X_{3}) h(X_{3}, X_{2})dX_{3}
\end{equation}

to produce the overall relation

\begin{equation}
y(X_{1}, X_{2}) = 1 + \frac{\rho}{2\pi}\int y(X_{1}, X_{3})f(X_{1}, X_{3}) \times \Big[ y(X_{3}, X_{2})f(X_{3}, X_{2}) + y(X_{3}, X_{2}) - 1 \Big] dX_{3}
\end{equation}


\subsubsection{Rose Potential Model}

The rose potential is another model first introduced by \cite{RoseOG}.
This model, while similar to the three-pronged BN2D, is notably different in that the rose potential model simplifies the model by use of a radial sinusoidal plot to make the three "prongs" of the particle. 

\subsubsection{Two-Dimensional Modeling}

Something other than OOPSE? (not seeing obvious answer other than "custom code modified/forked from existing 3D tools")

The Object Oriented Parallel Simulation Engine (OOPSE) was introduced by \cite{OOPSE} as a relatively light-weight molecular dynamics simulation package focused on "efficiently integrating equations of motion for atom types with orientational degrees of freedom" (from abstract).
While OOPSE was further developed and renamed OpenMD, a fork of OOPSE was developed specifically to model water in two dimensions.

\subsection{Literature Review on Relevant Works}









intro: Paragraph refreshing current work. Emphasize computational efficiency of Rose-Potential over Mercedez-Benz model.

\section{Goal of Project}

The objective of this work was to model two-dimensional water with a surface designed to discourage crystal growth at freezing temperatures. 
The design of the surface was the primary focus. 
Successfully designing a surface capable of discouraging water ice formation at freezing temperatures would provide valuable information in designing a three-dimensional model of the same type at a reduced computational cost. 

Idea: adjust freezing point depression to be freezing point modification.

\section{Tools and Terms}

Either a refresh from intro or a detailed explanation of OOPSE and the reduced terms.

\subsection{OOPSE in 2D}

Detail differences in computation system that Dr. Fennell developed to allow 2D MD(?) on a 3D program.

\subsection{Reduced Terms: 2D analogues}

Detail differences in dimensionality and define the reduced dimensions. Still working on the understanding/equations.

\section{Designing System}

Include: 
ensemble, 
thermodynamic variables, 
box attributes (size, pressure, temp, etc), 
number of waters,
surface (size, spacing between beads of surface, charges, LJ values, etc)

\subsection{Defining the Surface}

Explain how to develop the surface in the program and how to build a custom surface 
(walk through surface builder here)

\subsubsection{Manipulation of LJ Potential}

Manipulate $\sigma$ and $\epsilon$ values to effectively adjust the radius and interaction strength of surface beads.

\subsubsection{Manipulation of bead spacing}

Detail design of optimizing bead spacing for freezing encouragement or disruption.

\section{Results}

Success of freezing point elevation, pending results for freezing point depression

\subsection{Future Work}

Continued efforts to disrupt crystal growth on the surface. 


