\chapter{Sampling Conformation Landscapes by Rotatable Bond Degrees of Freedom}
\label{ch:ConformationLandscape}

\section{Introduction to Topic}

EACH SUBSECTION: DEFINITION OF TERMS

For ROUGHLY forty years, computational programs have allowed investigators to model chemical systems with high accuracy to determine their physical properties.

\subsection{A Brief History on Conformation Landscapes}

\subsubsection{Levinthal's Paradox}

Discuss history of Levinthal and his paradox. Provide the non-paradoxical solution. 

Next: Levinthal golf courses by Ken Dill.
\subsection{Literature Review on Relevant Works}











intro: 
Reference introduction: inherent complexity of "objective" or "exhaustive" search, levinthal's paradox as it applies to non-biological systems, ELSE?






Run through a set of dihedral positions at a constant interval. 
Selection of lowest-energy optimization organized on dihedral values. 
Quick determination of importance of dihedral based on how heavily it impacts internal energy. 
Splitting "best" dihedral into smaller interval to repeat the process. 

This method produces an interesting multilayered visual plot with a zooming effect toward the lowest energy conformer.
An example of how this might look for a two-dihedral molecule is given in figure \ref{fig:variableResolutionSample}.

\begin{figure}
	
	\centering
	
	\includegraphics[width=0.75\textwidth]{VRSGraphEX.jpeg}
	
	\caption{example VR chart (hand-drafted, CGI pending)}
	
	\label{fig:variableResolutionSample}
		
\end{figure}

\section{Design of System}

System designed in Python for ease of development and compiled via Cython for computational efficiency. 
Utilizes Gaussian and UCSF Chimera, but can be redesigned for any computational programs that accomplish the desired tasks.
An overview of system flow given in figure \ref{fig:VRSDesign}.


\begin{figure}
	\centering 
 		\makebox[\textwidth][c]{\includegraphics[width=1.3\textwidth]{AlgFlow.png}}
		\caption{Flow of method design for variable resolution conformation landscape search.}
		\label{fig:VRSDesign}

\end{figure}


This design, with implementation being a current work in progress, but should{\texttrademark} work as a cascade toward the lowest energy conformer in each case.

\subsection{Variation of Theory and Basis Set Usage by System Size and largest atom type}

System will have inherent restrictions. Give an example of large system with simple atoms, small system with complex atoms.
System estimates quantity and cost of calculations based on computational limits defined by user for various theory-basis set pairings. 
System optimizes calculations for the scale of run (is it the first broad-scope search, or a final near-exact search).

% Good to have: data on how theory and basis set alter computational requirement. Available in literature? Create data from runs?

\subsection{Computational Optimization by Varying Resolution}

Extant work not optimized for a general search (negative claim: make sure literature has nothing). 
Design should work with additional development (primary focus this semester) as a general search tool. 


\subsection{Inherent Complications}

Complications of size and atom type, impossible conformers, duplications, limited computational resources.

\section{Results}

Current success: finding accepted lowest energy conformer of a two-dihedral system by manually cranking each step. 
Self-running is still a work in progress.

\subsection{Problems}

Difficulty in defining an abstract system based on arbitrary hardware limitations. Propose a test-run to determine efficiency and resource availability.

\subsection{Anticipated Approaches for Future Work}

Putting system into a single cohesive program. Further optimizing Theory/Basis Set determination by computational efficiency as well as system size ( determine an upper-limit of computation?)