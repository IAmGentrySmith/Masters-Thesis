\chapter{Germanium Compounds and QM Concerns}
\label{ch:Germanium}

\section{Modeling Germanium Compounds}

EACH SUBSECTION: DEFINITION OF TERMS

This will be an interesting section as there is extremely little in terms of Ge computational work. Perhaps a broader search will yield interesting results. For sake of thoroughness, I will also include work on computational energy optimization in general and work through complications brought by the size of Ge. I might also include a portion on the statistical spread of conformations at a given temperature (internal energy?) I may include a sentence or paragraph on Gaussian-based publications. 

\subsection{Tools for Modeling Germanium}

Computational Requirements and reasons for those requirements.

Germanium is not the most-studied atom in computational works.
The majority of Germanium studies are done with Gaussian (citation needed?).

\subsection{Computational Complexity of Germanium Compounds}

Draw-backs of modeling Germanium. Uncommon but still necessary for wetwork.

\subsection{Literature Review on Relevant Works}

Make note of various Germanium modeling research. Make note of tools and methods used.









In the fall of 2016, the Fennell Group was contacted by Dr. Scott Weinert to study a germanium-based compound comprised of a backbone hexagermane chain, two phenyl rings off the internal atoms, and isopropyl groups on the terminal germanium atoms (hereafter referred to as "hexagermane"). 
Dr. Weinert's lab had received curious results from a prior collaborator and sought confirmation by a second computational group. 

\section{Current Methods}

Reiterate the scarcity of extant Germanium studies. Probably definitely reference Dr. Weinert's work.

\subsection{Precursory Work}

Dr. Weinert's lab had received curious results from a prior collaborator and sought confirmation by a second group. 

From the information provided in \ref{fig:Collab6GeOG}, the energy from Density Functional Theory calculations suggest that the most energetically-favorable conformation of hexagermane be the trans-cis-trans form. 

\begin{table}[]
	\centering
	\caption{Collaborator's Hexagermane Energies by Conformation \\ (Density Function Theory, Unknown Basis Set)}
	\label{fig:Collab6GeOG}
	\begin{tabular}{llll}
		\textbf{Conformation} & Energy ($E_{h}$)    & $\Delta$ Energy ($E_{h}$) & $\Delta$ Energy ($\frac{Kj}{mol}$) \\ \cline{2-4} 
		Trans-coplanar        & -15014.8403143 & 0.0066255            & 17.39525025                        \\
		Cis-Trans-Cis         & -15014.7983311 & 0.0486087            & 127.6221418                        \\
		Trans-Cis-Trans       & -15014.8469398 & 0.0000000            & 0                                  \\
		Cis-Trans-Trans       & -15014.8246918 & 0.0222480            & 58.412124                         
	\end{tabular}
\end{table}

\section{Modeling Germanium molecules in Gaussian}

This is the meat of the chapter - the Gaussian problem.

\subsection{Recognition of Problem}

Include Dr. Fennell's image. Relate to strangeness of collaborator's data. FRAME PROBLEM WITH THEORY AND BASIS SET

\subsection{Confirmation of Germanium consistency from other systems}

A minor note of confirmation of Gaussian problem from consistent results from other programs with same theory and basis set.

\subsection{Correcting the Issue}

Use basis set from Basis Set Exchange, detail selection and usage in input settings.