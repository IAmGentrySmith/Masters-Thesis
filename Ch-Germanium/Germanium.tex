\chapter{Germanium Compounds and QM Concerns}
\label{ch:Germanium}

\section{Modeling Germanium Compounds}

EACH SUBSECTION: DEFINITION OF TERMS

This will be an interesting section as there is extremely little in terms of Ge computational work. Perhaps a broader search will yield interesting results. For sake of thoroughness, I will also include work on computational energy optimization in general and work through complications brought by the size of Ge. I might also include a portion on the statistical spread of conformations at a given temperature (internal energy?) I may include a sentence or paragraph on Gaussian-based publications. 

\subsection{Tools for Modeling Germanium}

Computational Requirements and reasons for those requirements.

Germanium is not the most-studied atom in computational works.
The majority of Germanium studies are done with Gaussian (citation needed?).

\subsection{Computational Complexity of Germanium Compounds}

Draw-backs of modeling Germanium. Uncommon but still necessary for wetwork.

\subsection{Literature Review on Relevant Works}

Make note of various Germanium modeling research. Make note of tools and methods used.









\section{The Initial Problem: Germanium Study}

During Fall 2017, Dr. Christopher Fennell was approached by Dr. Charles Weinert of OSU to continue a collaborative effort in sampling conformation energies of two germanium-based compounds of interest to Dr. Weinert's work. 
Seen as an opportunity to train a new graduate student in conformational calculations, this project was delegated to me.
The initial focus was to create the two compounds in a 3D modeling program, save a file of each, run a conformation optimization program on a supercomputer, and read the output to report the findings.
As detailed below, this work led to impossibilities, curiosities, and inconsistencies that resulted in a general solution and a discovery of a flaw in a popular computational program.

\subsection{Parameters of Work and Previous Collaborator's Results}

The two subject germanium-based compounds are very similar: a germanium backbone with terminal isopropyl groups and internal phenyl rings. 
One compound constituted a pentagermanium chain while the other a hexagermanium backbone. 
The molecular formula for both is $Pr^{i}_{3}Ge(GePh_{2})_{n}GePr^{i}_{3}$ where n equals 3 for the pentagermane or 4 for the hexagermane compounds, respectively.
An example image of both compounds in their fully-trans configurations are provided in figures \ref{fig:Ge5TransAll} and \ref{fig:Ge6TransAll}.

\begin{figure}
	
	\centering
	
	\includegraphics[width=0.75\textwidth]{placeholder.png}
	
	\caption{Fully trans configuration of pentagermanium-based compound.}
	
	\label{fig:Ge5TransAll}
	
\end{figure}

\begin{figure}
	
	\centering
	
	\includegraphics[width=0.75\textwidth]{placeholder.png}
	
	\caption{Fully trans configuration of hexagermanium-based compound.}
	
	\label{fig:Ge6TransAll}
	
\end{figure}
\begin{table}[]
	
	\centering
	
	\begin{tabular}{llll}
		Conformation & Energy ($E_{h}$)    & $\Delta$ Energy ($E_{h}$) & $\Delta$ Energy ($\frac{KJ}{mol}$) \\ \cline{1-4} 
		Trans-coplanar        & -15014.8403143 & 0.0066255            & 17.39525025                        \\
		Cis-Trans-Cis         & -15014.7983311 & 0.0486087            & 127.6221418                        \\
		Trans-Cis-Trans       & -15014.8469398 & 0.0000000            & 0.0000000                                  \\
		Cis-Trans-Trans       & -15014.8246918 & 0.0222480            & 58.412124                         
	\end{tabular}
	
	\centering
	
	\caption{Collaborator's Hexagermane Energies by Conformation \\ (density functional theory, unknown basis set, energy in Hartrees and KJ/mol)}
	
	\label{tab:Ge6CollabEnergies}
	
\end{table}
Dr. Weinert had worked previously with an additional collaborator who provided conformation data supplied in table \ref{tab:Ge6CollabEnergies}.
If I want to cite somebody, all I do is type in the citation for \cite{BFIceOG}.


The approach of labeling the conformation shape of each compound, given the many points of torsion, focuses on the backbone structure. 
As the raw data from the collaborator was not available, the general dihedral angles of cis and trans proved a vexing focus for initial efforts at conformer design.
Using Newman projections like in figure 
% MAKE NEWMAN PROJECTION
REF!
% MAKE NEWMAN PROJECTION
as a visual guide, each Ge-Ge bond was defined as cis or trans based on the relative angle produced by the two  adjacent bonded Ge atoms to each subject Ge.
Specifically, the bonds are marked cis if the most acute angle is 90$^{\circ}$ or fewer, and likewise trans if greater than 90$^{\circ}$ up to the maximum 180$^{\circ}$.
Effectively the cis and trans angles coincide with gauche and anti in organic structure nomenclature.
Terminal germanium atoms are not considered as a part of the conformation state. 
This is partly due to the definition in labeling where the terminal germanium does not have an adjacent germanium for the measured relative angle, in addition to the assumed 
% CONFIRM GROUP THEORY - possibly S3?
C$_{3}$
% CONFIRM GROUP THEORY
symmetry of the terminal Ge with three isopropyl groups reducing the relative effects of terminal germanium rotation.
Effectively, only dihedrals formed by four consecutive Ge are given a cis or trans label.

\subsection{Design and Approach to Solution}

The initial approach involved an attempt at basic replication of the collaborative results.
This design gradually became more complex as I

\subsubsection{Design 1: Occam's Smallest Razer}

With each non-terminal Ge-Ge dihedral initially labeled cis or trans for 0$^{\circ}$ or 180$^{\circ}$, about 3 unique pentagermane and 6 unique hexagermane structures were built visually on a 3D visualization program (Avogadro).
These were rotated without consideration for the phenyl rings populating the non-terminal Ge atoms.
Each molecule was subjected to an energy minimzation in Gaussian 09 with the B3LYP hybrid function and STO-3G basis set as a single particle in a vacuum at otherwise default settings. 
See code in 
% REFERENCE EARLY G09 CODE
REF
% REFERENCE EARLY G09 CODE
for a sample job command file.

Unsurprisingly, only the fully trans conformers successfully converged (a 22\% success rate) into a stable form. 
These troubles were likely caused by the poor design of the initial conformers. 
With initial results, the conformer design was altered into a more systematic approach with some consideration for the phenyl rings.

\subsubsection{Design 2: A Blunt Effort}

In the second iteration of the conformer design process, a greater number of backbone conformers were generated. 
Instead of the simple 180$^{\circ}$ opposition between the cis and trans conformers, more intentional initial angles seen in Newman projections were selected.
Specifically, the anti and both gauche angles were chosen for the natural local minima in a non-bulky molecule, with both gauche angles (60 and 300) labeled as cis and the anti angle (180) as trans. 
For initial conformer design, these backbone angles were limited to three positions: 60$^{\circ}$, 180$^{\circ}$, or 300$^{\circ}$.
For the hexagermane compound, these structures were sequentially labeled trans-trans-trans, trans-trans-cis, trans-cis-trans, et cetera until all major unique conformers were produced.
For clarity, each conformer was identified by the dihedral angles (60-60-60, 60-60-180) in increasing order (Ge 1-2-3-4, Ge 2-3-4-5, Ge 3-4-5-6 dihedral).
The phenyl rings on the non-terminal Ge atoms were left untouched from an initial steepest-descent minimization available from Avogadro ran in the fully trans conformer.

To prevent potentially strong interactions between adjacent phenyl rings, an additional steepest-descent minimization from Avogadro was initially ran with the conformer-defining Ge-Ge dihedral angles locked in place. 
Additionally, a visual inspection of the phenyl rings and manual adjustments were utilized on Avogadro to reduce the chance of a relatively high energy local minima conformer. 
The phenyl rings usually were settled in a form of pi stacking or some kind of perpendicular ring interaction, based on relative energy stability according to the immediate simple minimization available. 

To further avoid backbone rotation restrictions, variations of the bulky molecules were also produced. 
These included versions where the phenyl rings were replaced by methyl groups and also where the isopropyl ends were additionally replaced by methyl groups. 
There intention in these designs were to observe the shift in relative energy between the sets of conformers to determine how significant of a role the phenyl rings and isopropyl groups played.
These variations, along with the original form structures, were subject to the same calculations as in the first design: Gaussian 09, B3LYP hybrid functional, STO-3G basis set, no angle restrictions, single particle in a vacuum, otherwise default parameters.
The results of these calculations are tabulated in table
% REF TABLE
REFNAME
% REF TABLE 

% USELESS DATA PUT HEAYAH

Immediately obvious in the table are the considerable number of nonconverged results. 
An unexpected bulkiness trend followed that a fully methylated variation of the structure was most likely to converge to a stable state, while the fully internal phenyl structures with methyl ends slightly reduced convergence and the original fully internal phenyl structures with isopropyl ends drastically reduced convergence.
The common-sense expectation that the addition of the phenyl ends would reduce stability was not realized in these results. 
A deeper exploration into the change of stability is a promising avenue for future investigation, but was not further explored in this work.
As
% IS IT HIGHLIGHTED???
higlighted
% IS IT???
in table 
%
REF,
%
the lowest energy conformer for each structure form varied greatly, but never included the fully trans conformer and only once the collaborator-reported trans-cis-trans conformer as the most stable.
Still, given the considerable amount of nonconverged conformers, a new design was necessary to further improve the scope of the lowest energy conformation search.

\subsubsection{Design 3: Death by 1.59 Million Cuts}

In the final version of the conformer generation effort, additional creation efforts were focused on the individual phenyl rings. 
The unfavorable interactions between the phenyl rings were considerable hurdle in the previous designs and a potential explanation for the large number of nonconverged structures, including the possibility that the terminal isopropyl hexagermane structures contained particularly unfavorable interactions among the phenyl rings.
This third design sought to remove the uncertainty in phenyl ring bulkiness by applying the same approach as the backbone generation: create unique conformers of every backbone torsion and phenyl ring, limiting each torsion to one of three rotational positions following the Newman projection style. 
Unfortunately, this task proved prohibitively large.

As an explanation for the insurmountability of the problem, consider the hexagermane structure. 
The germanium dihedrals represent three rotatable bonds each with three initial positions. 
To include the phenyl rings would require the inclusion of eight new rotatable bonds each with three initial positions.
Additionally, considering each terminal germanium's rotation while ignoring each isopropyl's rotatable bonds adds two initial positions each with three initial positions. 
Together, this creates a structure with 13 rotatable bonds each with three initial positions. The number of conformers follows as $3^{13} = 1,594,323$ initial conformers. 
Now we must consider the computational aspect of this many conformers.
At 10 conformers rotated and generated per second and 16 KB per conformer, the initial conformers would require 44.3 hours and generate 25.49 GB of data just in the initial structures.
At an average of 72 minutes per computation and 73.7 MB produced at B3LYP hybrid functional and STO-3G basis set and access to all 255 regular nodes of Oklahoma State University's Cowboy cluster running in parallel, the complete computation would generate 117.5 TB of data and require 312 days of continuous computation to determine a possible lowest energy conformer of this one molecule at a relatively low level basis set and theory.
A request to utilize 100\% of university supercomputer resources for nearly a year for the sake of determining the lowest energy conformer of one molecule would likely be rejected, so this task would likely require a time scale of years or even decades to produce with shared access to university resources. 
While conventionally considered a small molecule, the scale of conformers and computational requirements pushes this problem into the realm of Levinthal's paradox.

While this third design would have likely revealed the lowest energy conformer, or at least one considerably close the the exactly lowest energy conformer, the effort ultimate fails under its own weight.
Even with efforts to truncate duplicate forms, the problem of scale remains.
A reduction by 50\% still requires a computation effort in the timescale of years or decades for the calculation of a single molecule.
For an effective computational outlook, this system needs to be reduced by several orders of magnitude.

\subsection{Scale Reduction Efforts}

For a system with conformers on the millions scale and computations on the hour scale, a magnitude reduction in either aspect would improve the practicality of this design approach.
For example, by simplifying the computational method from 72 minutes on average to 5 minutes on average, the overall computational requirement would be reduced by 92\%, a full order of magnitude. 
Unfortunately, reducing the complexity of the method sacrifices the reliability of data.
A potential solution here would be to create rounds of calculations at different complexities, where each sequential round restricts the pool of potential conformers.
Ideally, the balance of the increasing computational complexity and the decreasing pool size would maintain a consistent computational requirement.
For example, a new round using a higher functional theory and basis set at 5x computational requirement would ideally be paired with a reduction in conformer pool size by a factor of 5.
This would produce a series of calculation sets with additive computational requirement instead of a magnitudinal expansion.

The natural next question lies within the reliability of basis sets and functional theories. 
It naturally follows that a less-accurate method should not be relied on while better methods exist. 
However, considering the scale of the conformer pool, it follows that a less accurate method would still produce energy values with a roughly similar internal consistency. 
For example, a 180-0-180 form of the hexagermane compound with parallel phenyl rings as modeled in figure
% FIGURE
REF
% FIGURE
will have intense syn interactions between some phenyl rings and will likely not yield a desirable energy value at any level of calculation while a fully trans form with perfect pi stacking phenyl rings will likely have a lower energy value at all levels of calculation.
It follows that, at lower levels of accuracy, the extremely high energy conformers can be pruned from the pool early and drastically reduce overall computational requirements.
A generic effort at producing a method in this style is detailed in chapter \ref{ch:ConformationLandscape}, while the remainder of this chapter details additional efforts of calculating these germanium compounds.

\subsection{Attempts at Simplification}

One potential avenue of simplifying the process is computing the energy minimizations of lower-period atoms (e.g. a carbon backbone instead of germanium) and then applying a correction factor for a net reduction in computation time.
As a period 4 element, germanium exhibits computational qualities similar to but more complicated than both carbon and silicon.
Using tested samples, an energy minimization of a carbon-backbone molecule instead of the germanium represented a 92\% reduction in computation speed.
Assuming a nominal correction factor exists and can be applied, this represents an order of magnitude reduction in computation time with one simplification. 
Potentially, this would allow investigators to much more quickly eliminate high energy conformers and more rapidly reduce the scope of the search.

The approach to acquiring sufficient data for a possible correction factor involved running an extremely simplified form of the germanium compounds, specifically a butagermane backbone with hydrogens occupying all terminal and internal bonds.
This reduced the complication and complexity of bulkiness and allowed for quick full torsion rotations about the single Ge-Ge-Ge-Ge dihedral.
By operating at intervals of 5$^{\circ}$, a full torsion drive provides a glimpse at relative energies of the molecule at 72 discrete states. 
Then, once multiple torsion drives had completed in multiple group four elements (isobutyl C, Si, and Ge were all built and tested), the energies could be compared and analyzed for any relative or absolute scaling at the additive or multiplicative reference. 

For a full comparative set, 3456 points of analyzed data were generated for each reference molecule's free energy in comparison with the others. 
Unsurprisingly, no simple correction factor arose by method of a simple additive or multiplicative term applied toward all torsion points. 
A future avenue of research would be to further explore this with depressive or polynomial terms to discover whether a simple corrective function might exist with specific molecules.

\subsection{Discovery of Program Flaw}

%asdf












