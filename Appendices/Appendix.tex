\chapter{Ice Ih to Ice XI Conversion}
\label{ch:App:CrystalDisorg}

Listed below is the source code utilized in the conversion of a .pdb Ice Ice I$\mathrm{_{h}}$ structure into an Ice XI structure. This code is functional in a Python 2.7 environment with the included packages: NumPy version 1.14.3 and SciPy version 1.1.0.

\section{Brief Sample of Ice XI .PDB File}
\lstinputlisting[firstline=1,lastline=30,breaklines=true]{codes/IceXI.pdb}

\section{Code: Crystal Disorganizer Tool}
\lstinputlisting[language=Python,breaklines=true]{codes/PDBDisorganize.py}

%% REMOVED - TOO MUCH INFO
%\section{PDB: Parent Ice XI}
%
%The following shows the parent Ice XI .pdb file used. Any .pdb file of an ice structure that follows the HETATM or ATOM style and the Bernal-Fowler ice rules should also work.
%\lstinputlisting[breaklines=true]{codes/IceXI.pdb}
%
%\section{PDB: Sample Generated Ice I$\mathrm{_{h}}$}
%
%The following details a generated Ice I$\mathrm{_{h}}$ .pdb file from the above parent.
%
%\lstinputlisting[breaklines=true]{codes/IceIh.pdb}

%% REMOVED: REMOVED CHAPTER FROM THESIS
%\chapter{Two-Dimensional Rose-Potential Water}
%\label{ch:App:2DWater}
%
%Below is the python script used to adjust the Rose-Potential system for various interfaces.
%
%\section{Code: Surface Adjustment Tool}
%This tool was created to adjust surface information in a given system.
%\lstinputlisting[language=Python,breaklines=true]{codes/oopseSurfaceMaker.py}
%
%% This is monstrous - unreasonable to include
%% \section{Surface Visualization Tool}
%% This tool was provided by Dr. Fennell to convert the completed OOPSE run into povre files to be converted into images or gifs.
%% \lstinputlisting[language=Perl,breaklines=true]{codes/surfaceRose2pov.pl}


\chapter{Germanium Landscape}
\label{ch:App:Germane}

\section{Sample Gaussian 09 Germanium File}
\label{SampleGeRunFile}
Command files like the one below were built using Dr. Fennell's Gaussian 09 run builder script and proved very effective in producing command files.
\lstinputlisting[breaklines=true]{codes/6geSample.cmd}

\section{Building Group 4 Chains}

While briefly mentioned and the subject of research for some time, the butyl-IV chain builder is detailed below. 
Ultimately unsuccessful in the initial trials, these scripts may serve a purpose in further work.

This first script builds a parent set of all possible C, Si, and Ge butylalkyl chains.

\lstinputlisting[language=Python,breaklines=true]{codes/grp4builder.py}

This second script takes the original trans-all butyl chain and enumerates 72 torsional rotations into a folder.

\lstinputlisting[language=Python,breaklines=true]{codes/ChimeraRot.py}

\section{Collecting and Comparing Torsional Data}

These two scripts were utilized to reduce the output data into an energy value with normalized intensity from 0 to 1. The third script compares two of these files and looks for any additive or multiplicative trend.

This first file reads energy data and creates a list of absolute energy values per torsion degree.
\lstinputlisting[language=Python,breaklines=true]{codes/4gepullStatPDBs.py}

This second file converts the first file into a relative scale from 0 to 1.
\lstinputlisting[language=Python,breaklines=true]{codes/4getreatAbsEnergies.py}

This third script compares two generated files using the prior scripts. It can compare the generated absolute energy with the relative energy files. It was often run as a loop through every permutation of the group 4 builder.

\lstinputlisting[language=Python,breaklines=true]{codes/compEnerData.py}


\chapter{Conformation Landscapes}
\label{ch:App:ConfLand}

Listed below are two example Germanium PDB files. The first is for the end-goal hexagermane in the trans-trans-trans conformation with isopropyl groups on the terminal Ge atoms. The second is for the simplified butagermane with fully protonated Germanium atoms.

\section{Code: hexagermane-transall.pdb}

\lstinputlisting[breaklines=true]{codes/hexagermane-transall.pdb}

The above molecule contains 154 atoms and 153 bonds, making it extremely computationally expensive for regular QM calculations. This made utilizing the large molecule as a trial system unreasonable due to the prohibitively long computation time for each conformation, assuming the conformation calculation would complete at all.

The below PDB file is the simplified butagermane with fully protonated Germanium atoms. As a significantly smaller system with only 14 atoms and 13 bonds, the relatively short computation time allowed the trial system to move with relative ease.

\section{Code: ge4h.pdb}

\lstinputlisting[breaklines=true]{codes/ge4h.pdb}

\section{Progress on Torsion Minimizer System}

While incomplete and largely nonfunctioning, this code is the current progress toward the implementation of the torsion minimizer system as outlined in \ref{fig:VRSDesign}.

\lstinputlisting[language=Python,breaklines=true]{codes/DTM/Runner.py}
